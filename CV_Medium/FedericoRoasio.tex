%%%%%%%%%%%%%%%%%%%%%%%%%%%%%%%%%%%%%%%%%
% "ModernCV" CV and Cover Letter
% LaTeX Template
% Version 1.1 (9/12/12)
%
% This template has been downloaded from:
% http://www.LaTeXTemplates.com
%
% Original author:
% Xavier Danaux (xdanaux@gmail.com)
%
% License:
% CC BY-NC-SA 3.0 (http://creativecommons.org/licenses/by-nc-sa/3.0/)
%
% Important note:
% This template requires the moderncv.cls and .sty files to be in the same 
% directory as this .tex file. These files provide the resume style and themes 
% used for structuring the document.
%
%%%%%%%%%%%%%%%%%%%%%%%%%%%%%%%%%%%%%%%%%

%----------------------------------------------------------------------------------------
%	PACKAGES AND OTHER DOCUMENT CONFIGURATIONS
%----------------------------------------------------------------------------------------

\documentclass[11pt,a4paper,sans]{moderncv} % Font sizes: 10, 11, or 12; paper sizes: a4paper, letterpaper, a5paper, legalpaper, executivepaper or landscape; font families: sans or roman

\moderncvstyle{classic} % CV theme - options include: 'casual' (default), 'classic', 'oldstyle' and 'banking'
\moderncvcolor{green} % CV color - options include: 'blue' (default), 'orange', 'green', 'red', 'purple', 'grey' and 'black'
\newcommand\mynobreakpar{\par\nobreak\@afterheading}
\usepackage{lipsum} % Used for inserting dummy 'Lorem ipsum' text into the template

\usepackage[scale=0.78]{geometry} % Reduce document margins
%\setlength{\hintscolumnwidth}{3cm} % Uncomment to change the width of the dates column
%\setlength{\makecvtitlenamewidth}{10cm} % For the 'classic' style, uncomment to adjust the width of the space allocated to your name
\usepackage[utf8]{inputenc}
%----------------------------------------------------------------------------------------
%	NAME AND CONTACT INFORMATION SECTION
%----------------------------------------------------------------------------------------

%\firstname{Federico} % Your first nameStufen A1
%\familyname{Roasio} % Your last name
% All information in this block is optional, comment out any lines you don't need
%\title{Curriculum Vitae}
%\age{24}
%\dni{34.239.624}
%\address{Jorge Newbery 281}{Florencio Varela, Buenos Aires}
%\mobile{15 3168 0608}
%\phone{(011) 4255 0067}
%\fax{(000) 111 1113}
%\email{federoasio@gmail.com}
%\homepage{staff.org.edu/~jsmith}{staff.org.edu/$\sim$jsmith} % The first argument is the url for the clickable link, the second argument is the url displayed in the template - this allows special characters to be displayed such as the tilde in this example
%\extrainfo{additional information}
%\photo[70pt][0.4pt]{pictures/picture} % The first bracket is the picture height, the second is the thickness of the frame around the picture (0pt for no frame)
%\quote{"A witty and playful quotation" - John Smith}
\makeatletter
\renewcommand*{\makecvtitle}{%
  % recompute lengths (in case we are switching from letter to resume, or vice versa)
  \recomputecvlengths%
  % optional detailed information box
  \newbox{\makecvtitledetailsbox}%
  \savebox{\makecvtitledetailsbox}{%
    \addressfont\color{color2}%
    \begin{tabular}[b]{@{}p{10cm}@{}}%
    \ifthenelse{\isundefined{\@extrainfo}}{}{\makenewline\@extrainfo\\}%
      \ifthenelse{\isundefined{\@addressstreet}}{}{\makenewline\addresssymbol\@addressstreet %
        \ifthenelse{\equal{\@addresscity}{}}{}{\makenewline\@addresscity}} % if \addresstreet is defined, \addresscity will always be defined but could be empty
      \ifthenelse{\isundefined{\@mobile}}{}{\makenewline\mobilesymbol\@mobile} %
      \ifthenelse{\isundefined{\@phone}}{}{\makenewline\phonesymbol\@phone}%
      \ifthenelse{\isundefined{\@fax}}{}{\makenewline\faxsymbol\@fax}%
      \ifthenelse{\isundefined{\@email}}{}{\makenewline\emailsymbol\emaillink{\@email}}%
      \ifthenelse{\isundefined{\@homepage}}{}{\makenewline\homepagesymbol\httplink{\@homepage}}%
    \end{tabular}
  }%
  % optional photo (pre-rendering)
  \newbox{\makecvtitlepicturebox}%
  \savebox{\makecvtitlepicturebox}{%
    \ifthenelse{\isundefined{\@photo}}%
    {}%
    {%
      \hspace*{2cm}%
      \color{color1}%
      \setlength{\fboxrule}{\@photoframewidth}%
      \ifdim\@photoframewidth=0pt%
        \setlength{\fboxsep}{0pt}\fi%
      \framebox{\includegraphics[width=\@photowidth]{\@photo}}}}%
  % name and title
  \newlength{\makecvtitledetailswidth}\settowidth{\makecvtitledetailswidth}{\usebox{\makecvtitledetailsbox}}%
  \newlength{\makecvtitlepicturewidth}\settowidth{\makecvtitlepicturewidth}{\usebox{\makecvtitlepicturebox}}%
  \ifthenelse{\lengthtest{\makecvtitlenamewidth=0pt}}% check for dummy value (equivalent to \ifdim\makecvtitlenamewidth=0pt)
    {\setlength{\makecvtitlenamewidth}{\textwidth-\makecvtitlepicturewidth
}}%
    {}%
 % \begin{minipage}[b]{\makecvtitlenamewidth}%
    \raggedright\namestyle{{\color{color2!50}\@firstname} {\color{color2}\@familyname}}\par\medskip%
    \ifthenelse{\equal{\@title}{}}{\vskip-25pt}{{\centering\titlestyle{\@title}\par}\medskip}%
  %\hfill%
  % detailed information
%  \llap{
%\namefont{\color{color2!50}\@firstname} {\color{color2}\@familyname}}
\usebox{\makecvtitledetailsbox}%
%\end{minipage}\ignorespaces%
%}% \llap is used to suppress the width of the box, allowing overlap if the value of makecvtitlenamewidth is forced
  % optional photo (rendering)
  \usebox{\makecvtitlepicturebox}\\[2.5em]%
  % optional quote
  \ifthenelse{\isundefined{\@quote}}%
    {}%
    {{\centering\begin{minipage}{\quotewidth}\centering\quotestyle{\@quote}\end{minipage}\\[2.5em]}}%
  \par}% to avoid weird spacing bug at the first section if no blank line is left after \makecvtitle
\makeatother

\photo[100pt]{pictures/picture}
\firstname{Federico Martin}
\familyname{Roasio}
\address{Uspallata 676 6A}{Ciudad Autónoma de Buenos Aires}    
\mobile{15 3168 0608}
\phone{(011) 4255 0067}                
\email{federoasio@gmail.com}            
\extrainfo{25 años\\ Argentino - DNI 34.239.624}
%----------------------------------------------------------------------------------------

\begin{document}

\makecvtitle % Print the CV title

%----------------------------------------------------------------------------------------
%	EDUCATION SECTION
%----------------------------------------------------------------------------------------

\section{Educación}

\cventry{2007--2013}{Ingeniero Electrónico}{Facultad de Ingeniería de la Universidad de Buenos Aires}{Buenos Aires}{\textit{Promedio -- 9.28}}{Elegible Medalla de Oro y Diploma de Honor (en trámite)}  % Arguments not required can be left empty
\cventry{2002--2006}{Técnico en Equipos e Instalaciones Electromecánicas}{\newline Instituto Técnico San Juan Bautista}{Buenos Aires}{\textit{Promedio -- 9.95}}{Abanderado}

\section{Tesis de Grado}

\cvitem{Título}{\emph{Desarrollo de un sistema de navegación integrado aplicado a vehículos aéreos no \mbox{tripulados}}}
\cvitem{Tutor}{Dr. Ing. Juan Ignacio Giribet}
\cvitem{Descripción}{Este trabajo de tesis consistió en el diseño, la simulación, la implementación y la validación de un sistema de navegación integrado INS/GPS aplicado a vehículos
aéreos no tripulados. El trabajo implicó el desarrollo de hardware y la implementación del sistema de software en C/C++ sobre un sistema con Linux embebido.
}

%----------------------------------------------------------------------------------------
%	WORK EXPERIENCE SECTION
%----------------------------------------------------------------------------------------

\section{Experiencia}
\cventry{2012 -- Presente}{Desarrollador, Investigador}{\textsc{CONAE-FIUBA}}{}{}{Desarrollador e investigador en el Grupo de Procesamiento de Señales, Identificación y Control.
\newline{}
Detalle de las tareas realizadas:
\begin{itemize}
\item Desarrollo de sistemas de navegación, guiado y control para cohetes suborbitales.
\item Programación en lenguaje C de los algoritmos, comunicación con sensores e integración del sistema completo.
\item Análisis matemático de algoritmos de navegación.
\item Post-procesamiento de datos en tierra para validación del funcionamiento del sistema.
\end{itemize}}

\cventry{2014 -- Presente}{Ingeniero de Proyectos de Robótica Industrial}{\textsc{Tenaris}}{}{}{Global Trainee en la división Global de Automatización y Proyectos Especiales de Tenaris
\newline{}
Detalle de las tareas realizadas:
\begin{itemize}
\item Ingeniería básica y de detalle de proyectos de robótica industrial.
\item Desarrollo de especificaciones técnicas, contacto con proveedores.
\item Integración de sectores de producción, mantenimiento y seguridad en el desarrollo de proyectos.
\item Global Trainee Program. Capacitación en Procesos Productivos, Calidad, Compras, Supply Chain, Ventas y Marketing, Management, Comunicación, Administración y Finanzas.
\end{itemize}}

\cventry{2013 -- 2014}{Desarrollador Web}{\textsc{Oneloop Digital Agency}}{}{}{
Oneloop es una start-up de rápido crecimiento que se dedica al desarrollo de aplicaciones web para el mercado inmobiliario.
\newline{}
Detalle de las tareas realizadas:
\begin{itemize}
\item Diseño de arquitectura de software.
\item Desarrollo de software del lado del cliente y del lado del servidor. 
\item Mantenimiento de base de datos.
\end{itemize}}

\cventry{2011--2013}{Docente, Investigador, Desarrollador}{\textsc{FIUBA}}{}{}{
Docente, investigador y desarrollador en el Laboratorio de Sistemas Embebidos de la Facultad
de Ingeniería de la Universidad de Buenos Aires.
\newline{}
Detalle de las tareas realizadas:
\begin{itemize}
\item Validación de firmware para NXP Semiconductors USA. Coordinación general del proyecto,
 desarrollo de tests de firmware y documentación.
\item Docente del curso de posgrado “Protocolos de comunicación en sistemas embebidos”. Clases teórico prácticas sobre protocolos de comunicación utilizados
en sistemas embebidos. 
\item Docente del workshop "Introducción a LPCXpresso" en el Simposio Argentino de Sistemas Embebidos.
\item Docente del curso de posgrado “Programación en Lenguaje C”. Clases teórico prácticas para alumnos de posgrado que buscan
introducirse a la programación de microcontroladores en este lenguaje.
\end{itemize}}

%------------------------------------------------
\cventry{2010-2013}{Colaborador}{\textsc{FIUBA}}{}{}{Colaborador de la materias Seminario de Sistemas Embebidos, Señales y Sistemas, Circuitos Electrónicos I.}

\cventry{2005-2012}{Docente}{Particular}{}{}{Dictado de clases particulares de Matemática, Física y Química,
para alumnos secundarios y universitarios.}
%----------------------------------------------------------------------------------------
%	LANGUAGES SECTION
%----------------------------------------------------------------------------------------

\section{Idiomas}

\cvitemwithcomment{Español}{Nativo}{}
\cvitemwithcomment{Inglés}{Avanzado}{First Certificate in English -- Grade A}
\cvitemwithcomment{Alemán}{Básico}{Nivel A1}
%----------------------------------------------------------------------------------------
%	COMPUTER SKILLS SECTION
%----------------------------------------------------------------------------------------
\section{Conocimientos Generales}

\cvitem{Básico}{Pascal, Visual Basic, Ruby, AutoCAD, CAD/CAM, PLC, Metodologías Ágiles}
\cvitem{Intermedio}{\textsc{python}, javascript, jQuery, MySQL, HTML5, CSS3, Linux Embebido, Diseño PCB}
\cvitem{Avanzado}{C/C++, PHP, Git, Mercurial, Microcontoladores línea ARM, Matlab/Octave, RTOS, \LaTeX}

%----------------------------------------------------------------------------------------
%	AWARDS SECTION
%----------------------------------------------------------------------------------------

\section{Distinciones}
\cvitem{2013}{2do Puesto, Premio Pre Ingenería 2013, Centro Argentino de Ingenieros}
\cvitem{2012}{Beca de viaje, alojamiento e inscripción al TI Tech Day São Paulo 2012.}
\cvitem{2012}{Certificado de reconocimiento de NXP, Microcontroller Business Lines.}
\cvitem{2006}{Abanderado del Instituto Técnico San Juan Bautista.}
\cvitem{2005}{XXII Olimpíada Matemática Argentina - Certamen Nacional - Aprobado.}
\cvitem{2004}{XXI Olimpíada Matemática Argentina - Certamen Nacional - Aprobado.}
\cvitem{2004-2006}{Convocado por la Olimpíada Matemática Argentina para los exámenes selectivos de la
Olimpíada Iberoamericana de Matemática y de la Olimpíada Internacional de Matemática.}
\cvitem{2003}{XX Olimpíada Matemática Argentina - Certamen Nacional - Aprobado.}
\cvitem{2003}{Mención Especial -- XX Olimpíada Matemática Argentina --  XI Certamen Provincial.}
\cvitem{2002}{Representante argentino en el XXIV Torneo Internacional de las Ciudades.}
\cvitem{2002}{Representante argentino en la VIII Olimpíada Iberoamericana de Mayo.}
\cvitem{2002}{Mención Especial -- XIX Olimpíada Matemática Argentina -- X Certamen Provincial.}
\cvitem{2001}{Mención Especial -- X Olimpíada Matemática Ñandú -- Certamen Nacional.}

%----------------------------------------------------------------------------------------
%	COURSES SECTION
%----------------------------------------------------------------------------------------

\section{Cursos}
\cvitem{2014}{Machine Learning -- Coursera.org}
\cvitem{2013}{Introducción a DSP}
\cvitem{}{- \slshape{Carrera de Especialización en Sistemas Embebidos, UBA. (res. 4916/12).}}
\cvitem{2012}{Introducción a Linux Embebido.}
\cvitem{}{- \slshape{Carrera de Especialización en Sistemas Embebidos, UBA. (res. 4916/12).}}
\cvitem{2012}{Texas Instrument Tech Day São Paulo.}
\cvitem{}{- \slshape{Sitara-Linux Board Port on Beaglebone.}}
\cvitem{}{- \slshape{Desenvolvendo com AM335x usando o Code Composer Studio e
BeagleBone.}}
\cvitem{2012}{Simposio Argentino de Sistemas Embebidos.}
\cvitem{}{- \slshape{Workshop de Linux Embebido.}}
\cvitem{2011}{Escuela Argentina de Micro-nanoelectrónica Tecnología y Aplicaciones.}{
\cvitem{}{- \slshape{Advanced Digital Design Track.}}
\cvitem{2011}{Simposio Argentino de Sistemas Embebidos.}
\cvitem{}{- \slshape{Workshop de programación en mbed.}}
\cvitem{2004-2006}{XII, XII, XIV Seminario Especial de Apoyo -- Olimpiada Matemática Argentina}
\cvitem{}{- \slshape{Seminarios a distancia teórico prácticos de duración anual para la resolución de
problemas matemáticos de nivel internacional. Tutora: Lic. Marta Lance.}}


%----------------------------------------------------------------------------------------
%	PAPERS SECTION
%----------------------------------------------------------------------------------------
\section{Publicaciones}
\cvitem{2012}{Implementación de una biblioteca para generar interfaces gráficas de usuario en
sistemas embebidos de bajo costo. ETC2012 (Costa Rica).}
\cvitem{2012}{Diseño e Implementación de un Cuadricóptero de Vuelo Autónomo, CASE 2012.}
\cvitem{2012}{Interfaces gráficas de usuario en sistemas embebidos con sistemas operativos de
tiempo real. CASE 2012.}
\cvitem{2011}{Graphical User Interface for LPC17XX, Nota de aplicación para NXP Semiconductors.}
FIUBA
%----------------------------------------------------------------------------------------
%	INTERESTS SECTION
%----------------------------------------------------------------------------------------

\section{Intereses}

\renewcommand{\listitemsymbol}{-~} % Changes the symbol used for lists

\cvlistdoubleitem{Fútbol}{Tenis}
\cvlistdoubleitem{Turismo}{Economía}


%----------------------------------------------------------------------------------------
%	COVER LETTER
%----------------------------------------------------------------------------------------

% To remove the cover letter, comment out this entire block

%\clearpage

%\recipient{HR Departmnet}{Corporation\\123 Pleasant Lane\\12345 City, State} % Letter recipient
%\date{\today} % Letter date
%\opening{Dear Sir or Madam,} % Opening greeting
%\closing{Sincerely yours,} % Closing phrase
%\enclosure[Attached]{curriculum vit\ae{}} % List of enclosed documents

%\makelettertitle % Print letter title

%\lipsum[1-3] % Dummy text

%\makeletterclosing % Print letter signature

%----------------------------------------------------------------------------------------

\end{document}