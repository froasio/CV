%%%%%%%%%%%%%%%%%%%%%%%%%%%%%%%%%%%%%%%%%
% "ModernCV" CV and Cover Letter
% LaTeX Template
% Version 1.1 (9/12/12)
%
% This template has been downloaded from:
% http://www.LaTeXTemplates.com
%
% Original author:
% Xavier Danaux (xdanaux@gmail.com)
%
% License:
% CC BY-NC-SA 3.0 (http://creativecommons.org/licenses/by-nc-sa/3.0/)
%
% Important note:
% This template requires the moderncv.cls and .sty files to be in the same 
% directory as this .tex file. These files provide the resume style and themes 
% used for structuring the document.
%
%%%%%%%%%%%%%%%%%%%%%%%%%%%%%%%%%%%%%%%%%

%----------------------------------------------------------------------------------------
%	PACKAGES AND OTHER DOCUMENT CONFIGURATIONS
%----------------------------------------------------------------------------------------

\documentclass[11pt,a4paper,sans]{moderncv} % Font sizes: 10, 11, or 12; paper sizes: a4paper, letterpaper, a5paper, legalpaper, executivepaper or landscape; font families: sans or roman

\moderncvstyle{casual} % CV theme - options include: 'casual' (default), 'classic', 'oldstyle' and 'banking'
\moderncvcolor{green} % CV color - options include: 'blue' (default), 'orange', 'green', 'red', 'purple', 'grey' and 'black'
\newcommand\mynobreakpar{\par\nobreak\@afterheading}
\usepackage{lipsum} % Used for inserting dummy 'Lorem ipsum' text into the template

\usepackage[scale=0.75]{geometry} % Reduce document margins
%\setlength{\hintscolumnwidth}{3cm} % Uncomment to change the width of the dates column
%\setlength{\makecvtitlenamewidth}{10cm} % For the 'classic' style, uncomment to adjust the width of the space allocated to your name
\usepackage[utf8]{inputenc}
%----------------------------------------------------------------------------------------
%	NAME AND CONTACT INFORMATION SECTION
%----------------------------------------------------------------------------------------

\firstname{Federico Martin} % Your first name
\familyname{Roasio} % Your last name

% All information in this block is optional, comment out any lines you don't need
\title{Curriculum Vitae}
\address{Jorge Newbery 281}{Florencio Varela, Buenos Aires}
\mobile{15 3168 0608}
\phone{(011) 4255 0067}
%\fax{(000) 111 1113}
\email{froasio@fi.uba.ar}
%\homepage{staff.org.edu/~jsmith}{staff.org.edu/$\sim$jsmith} % The first argument is the url for the clickable link, the second argument is the url displayed in the template - this allows special characters to be displayed such as the tilde in this example
%\extrainfo{additional information}
\photo[70pt][0.4pt]{pictures/picture} % The first bracket is the picture height, the second is the thickness of the frame around the picture (0pt for no frame)
%\quote{"A witty and playful quotation" - John Smith}

%----------------------------------------------------------------------------------------

\begin{document}

\makecvtitle % Print the CV title

%----------------------------------------------------------------------------------------
%	EDUCATION SECTION
%----------------------------------------------------------------------------------------

\section{Educación}

\cventry{2007--2013}{Ingeniero Electrónico}{Facultad de Ingeniería de la Universidad de Buenos Aires}{Buenos Aires}{\textit{Promedio -- 9.28}}{Diploma de Honor (Título en trámite)}  % Arguments not required can be left empty
\cventry{2002--2006}{Técnico en Equipos e Instalaciones Electromecánicas}{\newline Instituto Técnico San Juan Bautista}{Buenos Aires}{\textit{Promedio -- 9.95}}{Abanderado}

\section{Tesis de Grado}

\cvitem{Título}{\emph{Desarrollo de un sistema de navegación integrado aplicado a vehículos aéreos no tripulados}}
\cvitem{Tutor}{Dr. Ing. Juan Ignacio Giribet}
\cvitem{Descripción}{Este trabajo de tesis consistió en el diseño, la simulación, la implementación y la validación de un sistema de navegación integrado INS/GPS aplicado a vehículos
aéreos no tripulados. El trabajo implicó el desarrollo de hardware y la implementación del sistema de software en C/C++ sobre un sistema con Linux embebido.
}

%----------------------------------------------------------------------------------------
%	WORK EXPERIENCE SECTION
%----------------------------------------------------------------------------------------

\section{Experiencia}

\cventry{2011--Presente}{Docente, Investigador, Desarrollador}{\textsc{FIUBA}}{}{}{
Docente, investigador y desarrollador en el Laboratorio de Sistemas Embebidos de la Facultad de Ingeniería de la Universidad de Buenos Aires. 
Mis principales tareas son el dictado de clases en cursos de grado y posgrado de programación orientada a sistemas embebidos, la publicación de trabajos científicos y 
la realización de trabajos de consultoría para empresas de la Argentina y el exterior.
\newline{}\newline{}
Detalle de las tareas realizadas:
\begin{itemize}
\item Validación de firmware para NXP Semiconductors USA. A cargo de la coordinación general del proyecto,
 desarrollo de tests de firmware y documentación.
\item Docente del curso de grado “Seminario de Sistemas Embebidos”. Clases prácticas de programación en C de microcontroladores de
la línea Cortex-M3 de ARM. 
\item Docente del curso de posgrado “Protocolos de comunicación en sistemas embebidos”. Clases teórico prácticas sobre protocolos de comunicación utilizados
en sistemas embebidos.
\end{itemize}}
\cventry{}{}{}{}{}{
\begin{itemize}
\item Docente del workshop "Introducción a LPCXpresso" en el Simposio Argentino de Sistemas Embebidos.
\item Docente del curso de posgrado “Programación en Lenguaje C”. Clases teórico prácticas para alumnos de posgrado que buscan
introducirse a la programación de microcontroladores en este lenguaje. 
\end{itemize}}

%------------------------------------------------
\cventry{2012 -- Presente}{Desarrollador, Investigador}{\textsc{FIUBA}}{}{}{Desarrollador e investigador en el Grupo de Procesamiento de Señales, Identificación y Control en el área
de Sistemas de Navegación, Guiado y Control orientado a aplicaciones aeroespaciales.}
\cventry{2005-Presente}{Docente}{Particular}{}{}{Me desempeño en el dictado de clases particulares de Matemática, Física y Química,
para alumnos secundarios y universitarios.}
\cventry{2011}{Colaborador}{\textsc{FIUBA}}{}{}{Colaborador de la materia Circuitos Electrónicos I.}
\cventry{2010-2011}{Colaborador}{\textsc{FIUBA}}{}{}{Colaborador de la materia Señales y Sistemas.}


%------------------------------------------------

%----------------------------------------------------------------------------------------
%	AWARDS SECTION
%----------------------------------------------------------------------------------------

\section{Distinciones}
\cvitem{2012}{Beca de viaje, alojamiento e inscripción al TI Tech Day São Paulo 2012.}
\cvitem{2012}{Certificado de reconocimiento de NXP, Microcontroller Business Lines.}
\cvitem{2006}{Abanderado del Instituto Técnico San Juan Bautista.}
\cvitem{2005}{XXII Olimpíada Matemática Argentina - Certamen Nacional - Aprobado.}
\cvitem{2004}{XXI Olimpíada Matemática Argentina - Certamen Nacional - Aprobado.}
\cvitem{2004-2006}{Convocado por la Olimpíada Matemática Argentina para los exámenes selectivos de la
Olimpíada Iberoamericana de Matemática y de la Olimpíada Internacional de Matemática.}
\cvitem{2003}{XX Olimpíada Matemática Argentina - Certamen Nacional - Aprobado.}
\cvitem{2003}{Mención Especial -- XX Olimpíada Matemática Argentina --  XI Certamen Provincial.}
\cvitem{2002}{Representante argentino en el XXIV Torneo Internacional de las Ciudades.}
\cvitem{2002}{Representante argentino en la VIII Olimpíada Iberoamericana de Mayo.}
\cvitem{2002}{Mención Especial -- XIX Olimpíada Matemática Argentina -- X Certamen Provincial.}
\cvitem{2001}{Mención Especial -- X Olimpíada Matemática Ñandú -- Certamen Nacional.}
%----------------------------------------------------------------------------------------
%	COMPUTER SKILLS SECTION
%----------------------------------------------------------------------------------------
\newpage
\section{Conocimientos Generales}

\cvitem{Básico}{Pascal, Visual Basic, CAD, CAM, PLC}
\cvitem{Intermedio}{\textsc{python}, \LaTeX, Office, OpenOffice, Windows, Linux Embebido, \newline Diseño PCB (Eagle), Git, Mercurial}
\cvitem{Avanzado}{C/C++, Microcontoladores línea ARM, Matlab, Octave, RTOS}

%----------------------------------------------------------------------------------------
%	LANGUAGES SECTION
%----------------------------------------------------------------------------------------

\section{Idiomas}

\cvitemwithcomment{Español}{Nativo}{}
\cvitemwithcomment{Inglés}{Avanzado}{First Certificate in English -- Grade A}

%----------------------------------------------------------------------------------------
%	COURSES SECTION
%----------------------------------------------------------------------------------------

\section{Cursos}
\cvitem{2012}{Introducción a Linux Embebido.}
\cvitem{}{- \slshape{Carrera de Especialización en Sistemas Embebidos, UBA. (res. 4916/12).}}
\cvitem{2012}{Texas Instrument Tech Day São Paulo.}
\cvitem{}{- \slshape{Sitara-Linux Board Port on Beaglebone.}}
\cvitem{}{- \slshape{Desenvolvendo com AM335x usando o Code Composer Studio e
BeagleBone.}}
\cvitem{2012}{Simposio Argentino de Sistemas Embebidos.}
\cvitem{}{- \slshape{Workshop de Linux Embebido.}}
\cvitem{2011}{Escuela Argentina de Micro-nanoelectrónica Tecnología y Aplicaciones.}{
\cvitem{}{- \slshape{Advanced Digital Design Track.}}
\cvitem{2011}{Simposio Argentino de Sistemas Embebidos.}
\cvitem{}{- \slshape{Workshop de programación en mbed.}}
\cvitem{2004-2006}{XII, XII, XIV Seminario Especial de Apoyo -- Olimpiada Matemática Argentina}
\cvitem{}{- \slshape{Seminarios a distancia teórico prácticos de duración anual para la resolución de
problemas matemáticos de nivel internacional. Tutora: Lic. Marta Lance.}}


%----------------------------------------------------------------------------------------
%	PAPERS SECTION
%----------------------------------------------------------------------------------------
\section{Publicaciones}
\cvitem{2012}{Implementación de una biblioteca para generar interfaces gráficas de usuario en
sistemas embebidos de bajo costo. ETC2012 (Costa Rica).}
\cvitem{2012}{Diseño e Implementación de un Cuadricóptero de Vuelo Autónomo, CASE 2012.}
\cvitem{2012}{Interfaces gráficas de usuario en sistemas embebidos con sistemas operativos de
tiempo real. CASE 2012.}
\cvitem{2011}{Graphical User Interface for LPC17XX, Nota de aplicación para NXP Semiconductors.}
\newpage

%----------------------------------------------------------------------------------------
%	ANALITICO
%----------------------------------------------------------------------------------------

\section{Analítico Universitario}

\cvlistdoubleitem{Introducción al Conocimiento de la Sociedad y el Estado: 10
}{Análisis Matemático I: 10}
\cvlistdoubleitem{Física: 10}{Química: 10}
\cvlistdoubleitem{Introducción al Pensamiento Científico: 9}{Álgebra Lineal I: 9}
\cvlistdoubleitem{Física I: 10}{Análisis Matemático II: 10}
\cvlistdoubleitem{Álgebra Lineal II: 10}{Análisis Matemático III A: 10}
\cvlistdoubleitem{Algoritmos y Programación I: 10}{Física II: 9}
\cvlistdoubleitem{Química I: 8}{Física III A: 10}
\cvlistdoubleitem{Técnicas Digitales: 10}{Laboratorio: 9}
\cvlistdoubleitem{Probabilidad y Estadística B: 6}{Análisis de Circuitos: 9}
\cvlistdoubleitem{Electromagnetismo: 9}{Matemática Discreta: 9}
\cvlistdoubleitem{Dispositivos Electrónicos: 8}{Señales y Sistemas: 10}
\cvlistdoubleitem{Laboratorio de Microcontroladores: 9}{Algoritmos y Programación II: 9}
\cvlistdoubleitem{Circuitos Electrónicos I: 10}{Teoría de Control I: 10}
\cvlistdoubleitem{Procesos Estocásticos: 9}{Seminario de Sistemas Embebidos: 10}
\cvlistdoubleitem{Seminario de Sistemas Embebidos: 10}{Circuitos Electrónicos II: 10}
\cvlistdoubleitem{Comunicaciones Analógicas y Digitales: 9}{Procesamiento de Señales I: 10}
\cvlistdoubleitem{Introducción a Proyectos: 9}{Instrumentos Electrónicos: 5}
\cvlistdoubleitem{Seminario de Accionamientos Variables: 8}{Comunicación de Datos: 9}
\cvlistdoubleitem{Procesamiento de Señales II: 10}{Legislación y Ejercicio Profesional de la Ingeniería Electrónica: 9}
\cvlistdoubleitem{Teoría de Control II: 10}{Tesis de Grado: 10}

%----------------------------------------------------------------------------------------
%	INTERESTS SECTION
%----------------------------------------------------------------------------------------

\section{Intereses}

\renewcommand{\listitemsymbol}{-~} % Changes the symbol used for lists

\cvlistdoubleitem{Fútbol}{Tenis}
\cvlistdoubleitem{Turismo}{Economía}


%----------------------------------------------------------------------------------------
%	COVER LETTER
%----------------------------------------------------------------------------------------

% To remove the cover letter, comment out this entire block

%\clearpage

%\recipient{HR Departmnet}{Corporation\\123 Pleasant Lane\\12345 City, State} % Letter recipient
%\date{\today} % Letter date
%\opening{Dear Sir or Madam,} % Opening greeting
%\closing{Sincerely yours,} % Closing phrase
%\enclosure[Attached]{curriculum vit\ae{}} % List of enclosed documents

%\makelettertitle % Print letter title

%\lipsum[1-3] % Dummy text

%\makeletterclosing % Print letter signature

%----------------------------------------------------------------------------------------

\end{document}